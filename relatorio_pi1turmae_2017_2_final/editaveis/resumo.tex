\begin{resumo}
Este primeiro ponto de controle traz uma introdução teórica a respeito do aparelho de ultrassom para qual fim ele é utilizado e os motivos e as utilidades de realizar esse exame de forma controlada a distância. Ele traz também a ideia inicial de como esse projeto pode ser implementado, traçando os requisitos necessários, a metodologia de trabalho da equipe, mostrando a integração entre as cinco engenharias presentes no projeto, as ferramentas de trabalho e comunicação do grupo, um macro cronograma das etapas do projeto até a sua conclusão, referências de dados coletados como ajuda inicial, o escopo do projeto (EAP) e por fim a solução visualizada pela equipe para o projeto proposto.  

 \vspace{\onelineskip}
    
 \noindent
 \textbf{Palavras-chaves}: projeto. ferramentas. cronograma e escopo.
\end{resumo}
